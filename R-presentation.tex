% Options for packages loaded elsewhere
\PassOptionsToPackage{unicode}{hyperref}
\PassOptionsToPackage{hyphens}{url}
%
\documentclass[
  ignorenonframetext,
]{beamer}
\usepackage{pgfpages}
\setbeamertemplate{caption}[numbered]
\setbeamertemplate{caption label separator}{: }
\setbeamercolor{caption name}{fg=normal text.fg}
\beamertemplatenavigationsymbolsempty
% Prevent slide breaks in the middle of a paragraph
\widowpenalties 1 10000
\raggedbottom
\setbeamertemplate{part page}{
  \centering
  \begin{beamercolorbox}[sep=16pt,center]{part title}
    \usebeamerfont{part title}\insertpart\par
  \end{beamercolorbox}
}
\setbeamertemplate{section page}{
  \centering
  \begin{beamercolorbox}[sep=12pt,center]{part title}
    \usebeamerfont{section title}\insertsection\par
  \end{beamercolorbox}
}
\setbeamertemplate{subsection page}{
  \centering
  \begin{beamercolorbox}[sep=8pt,center]{part title}
    \usebeamerfont{subsection title}\insertsubsection\par
  \end{beamercolorbox}
}
\AtBeginPart{
  \frame{\partpage}
}
\AtBeginSection{
  \ifbibliography
  \else
    \frame{\sectionpage}
  \fi
}
\AtBeginSubsection{
  \frame{\subsectionpage}
}
\usepackage{amsmath,amssymb}
\usepackage{iftex}
\ifPDFTeX
  \usepackage[T1]{fontenc}
  \usepackage[utf8]{inputenc}
  \usepackage{textcomp} % provide euro and other symbols
\else % if luatex or xetex
  \usepackage{unicode-math} % this also loads fontspec
  \defaultfontfeatures{Scale=MatchLowercase}
  \defaultfontfeatures[\rmfamily]{Ligatures=TeX,Scale=1}
\fi
\usepackage{lmodern}
\ifPDFTeX\else
  % xetex/luatex font selection
\fi
% Use upquote if available, for straight quotes in verbatim environments
\IfFileExists{upquote.sty}{\usepackage{upquote}}{}
\IfFileExists{microtype.sty}{% use microtype if available
  \usepackage[]{microtype}
  \UseMicrotypeSet[protrusion]{basicmath} % disable protrusion for tt fonts
}{}
\makeatletter
\@ifundefined{KOMAClassName}{% if non-KOMA class
  \IfFileExists{parskip.sty}{%
    \usepackage{parskip}
  }{% else
    \setlength{\parindent}{0pt}
    \setlength{\parskip}{6pt plus 2pt minus 1pt}}
}{% if KOMA class
  \KOMAoptions{parskip=half}}
\makeatother
\usepackage{xcolor}
\newif\ifbibliography
\usepackage{color}
\usepackage{fancyvrb}
\newcommand{\VerbBar}{|}
\newcommand{\VERB}{\Verb[commandchars=\\\{\}]}
\DefineVerbatimEnvironment{Highlighting}{Verbatim}{commandchars=\\\{\}}
% Add ',fontsize=\small' for more characters per line
\usepackage{framed}
\definecolor{shadecolor}{RGB}{248,248,248}
\newenvironment{Shaded}{\begin{snugshade}}{\end{snugshade}}
\newcommand{\AlertTok}[1]{\textcolor[rgb]{0.94,0.16,0.16}{#1}}
\newcommand{\AnnotationTok}[1]{\textcolor[rgb]{0.56,0.35,0.01}{\textbf{\textit{#1}}}}
\newcommand{\AttributeTok}[1]{\textcolor[rgb]{0.13,0.29,0.53}{#1}}
\newcommand{\BaseNTok}[1]{\textcolor[rgb]{0.00,0.00,0.81}{#1}}
\newcommand{\BuiltInTok}[1]{#1}
\newcommand{\CharTok}[1]{\textcolor[rgb]{0.31,0.60,0.02}{#1}}
\newcommand{\CommentTok}[1]{\textcolor[rgb]{0.56,0.35,0.01}{\textit{#1}}}
\newcommand{\CommentVarTok}[1]{\textcolor[rgb]{0.56,0.35,0.01}{\textbf{\textit{#1}}}}
\newcommand{\ConstantTok}[1]{\textcolor[rgb]{0.56,0.35,0.01}{#1}}
\newcommand{\ControlFlowTok}[1]{\textcolor[rgb]{0.13,0.29,0.53}{\textbf{#1}}}
\newcommand{\DataTypeTok}[1]{\textcolor[rgb]{0.13,0.29,0.53}{#1}}
\newcommand{\DecValTok}[1]{\textcolor[rgb]{0.00,0.00,0.81}{#1}}
\newcommand{\DocumentationTok}[1]{\textcolor[rgb]{0.56,0.35,0.01}{\textbf{\textit{#1}}}}
\newcommand{\ErrorTok}[1]{\textcolor[rgb]{0.64,0.00,0.00}{\textbf{#1}}}
\newcommand{\ExtensionTok}[1]{#1}
\newcommand{\FloatTok}[1]{\textcolor[rgb]{0.00,0.00,0.81}{#1}}
\newcommand{\FunctionTok}[1]{\textcolor[rgb]{0.13,0.29,0.53}{\textbf{#1}}}
\newcommand{\ImportTok}[1]{#1}
\newcommand{\InformationTok}[1]{\textcolor[rgb]{0.56,0.35,0.01}{\textbf{\textit{#1}}}}
\newcommand{\KeywordTok}[1]{\textcolor[rgb]{0.13,0.29,0.53}{\textbf{#1}}}
\newcommand{\NormalTok}[1]{#1}
\newcommand{\OperatorTok}[1]{\textcolor[rgb]{0.81,0.36,0.00}{\textbf{#1}}}
\newcommand{\OtherTok}[1]{\textcolor[rgb]{0.56,0.35,0.01}{#1}}
\newcommand{\PreprocessorTok}[1]{\textcolor[rgb]{0.56,0.35,0.01}{\textit{#1}}}
\newcommand{\RegionMarkerTok}[1]{#1}
\newcommand{\SpecialCharTok}[1]{\textcolor[rgb]{0.81,0.36,0.00}{\textbf{#1}}}
\newcommand{\SpecialStringTok}[1]{\textcolor[rgb]{0.31,0.60,0.02}{#1}}
\newcommand{\StringTok}[1]{\textcolor[rgb]{0.31,0.60,0.02}{#1}}
\newcommand{\VariableTok}[1]{\textcolor[rgb]{0.00,0.00,0.00}{#1}}
\newcommand{\VerbatimStringTok}[1]{\textcolor[rgb]{0.31,0.60,0.02}{#1}}
\newcommand{\WarningTok}[1]{\textcolor[rgb]{0.56,0.35,0.01}{\textbf{\textit{#1}}}}
\setlength{\emergencystretch}{3em} % prevent overfull lines
\providecommand{\tightlist}{%
  \setlength{\itemsep}{0pt}\setlength{\parskip}{0pt}}
\setcounter{secnumdepth}{-\maxdimen} % remove section numbering
\ifLuaTeX
  \usepackage{selnolig}  % disable illegal ligatures
\fi
\usepackage{bookmark}
\IfFileExists{xurl.sty}{\usepackage{xurl}}{} % add URL line breaks if available
\urlstyle{same}
\hypersetup{
  pdftitle={An Introduction to R},
  pdfauthor={Sarah Teichman},
  hidelinks,
  pdfcreator={LaTeX via pandoc}}

\title{An Introduction to R}
\subtitle{The basics}
\author{Sarah Teichman}
\date{}

\begin{document}
\frame{\titlepage}

\begin{frame}{What is R}
\phantomsection\label{what-is-r}
R is a programming language commonly used for data analysis and
statistics.

\begin{itemize}
\tightlist
\item
  reproducible
\item
  free
\item
  open-source
\item
  large community of users and developers
\end{itemize}

Download \href{https://cran.rstudio.com/}{here}.
\end{frame}

\begin{frame}{What is RStudio}
\phantomsection\label{what-is-rstudio}
RStudio is an Integrated Development Environment (IDE) for R.

\begin{itemize}
\tightlist
\item
  write code
\item
  run code
\item
  navigate files
\item
  visualize plots
\item
  open help files
\end{itemize}

Download \href{https://posit.co/download/rstudio-desktop/}{here}.
\end{frame}

\begin{frame}{How to access RStudio}
\phantomsection\label{how-to-access-rstudio}
\begin{itemize}
\tightlist
\item
  locally, if you have downloaded R and RStudio on your computer

  \begin{itemize}
  \tightlist
  \item
    uses your own computer and with access to your files
  \item
    has the computational resources and limitations of your computer
  \end{itemize}
\item
  remotely, via RStudio server

  \begin{itemize}
  \tightlist
  \item
    access via a web browser
  \item
    often available through your institution
  \item
    may have computational advantages
  \item
    useful in a course setting like this one!
  \end{itemize}
\end{itemize}
\end{frame}

\begin{frame}{How to access RStudio server}
\phantomsection\label{how-to-access-rstudio-server}
\begin{itemize}
\tightlist
\item
  find your personal R Studio server link from
  \href{https://hackmd.io/oz5sTY9KRCqdHHkM9iNJyg?view}{this spreadsheet}
  *update!!
\item
  username: stamps
\item
  password: stamps2024
\end{itemize}
\end{frame}

\begin{frame}{RStudio organization}
\phantomsection\label{rstudio-organization}
RStudio has a four pane layout.

\begin{itemize}
\tightlist
\item
  console (run single lines of code)
\item
  editor (open and write scripts)
\item
  environment etc. (see what objects exist in work space)
\item
  files etc. (navigate files, view plots, open help files)
\end{itemize}
\end{frame}

\begin{frame}[fragile]{Console}
\phantomsection\label{console}
Use the \textbf{console} to run individual lines of code.

\begin{Shaded}
\begin{Highlighting}[]
\DecValTok{5} \SpecialCharTok{+} \DecValTok{384}
\end{Highlighting}
\end{Shaded}

\begin{verbatim}
## [1] 389
\end{verbatim}

\begin{Shaded}
\begin{Highlighting}[]
\NormalTok{x }\OtherTok{\textless{}{-}} \DecValTok{10} \CommentTok{\# set variable with \textless{}{-} operator :) }
\NormalTok{y }\OtherTok{=} \DecValTok{6} \CommentTok{\# set variable with = operator :( }
\NormalTok{x }\SpecialCharTok{+}\NormalTok{ y}
\end{Highlighting}
\end{Shaded}

\begin{verbatim}
## [1] 16
\end{verbatim}
\end{frame}

\begin{frame}[fragile]{Editor}
\phantomsection\label{editor}
Use the \textbf{editor} for opening and writing scripts.

\begin{itemize}
\tightlist
\item
  for a workflow to be reproducible, all code should be written in a
  script (not in the console)
\item
  in R you are working in a folder on your computer

  \begin{itemize}
  \tightlist
  \item
    \texttt{getwd()} to see (get) your working directory
  \item
    \texttt{setwd()} to change (set) your working directory
  \end{itemize}
\item
  run code with \texttt{Run} button (and options) or
  \texttt{Ctrl}/\texttt{Command} + \texttt{Enter} for a single line
\end{itemize}
\end{frame}

\begin{frame}{Environment and History}
\phantomsection\label{environment-and-history}
\begin{itemize}
\tightlist
\item
  each object saved in your working space will be in the
  \textbf{environment}
\item
  \textbf{history} saves most recent lines of code
\item
  extension: you can add a \textbf{Git} plug-in to this pane for version
  control through GitHub

  \begin{itemize}
  \tightlist
  \item
    \href{https://happygitwithr.com/index.html}{here} is a great
    resource for R and Git!
  \end{itemize}
\end{itemize}
\end{frame}

\begin{frame}[fragile]{Files/Plots/Help}
\phantomsection\label{filesplotshelp}
\begin{itemize}
\tightlist
\item
  use files to navigate files on your computer
\item
  use plots to display visualizations
\item
  use help to access help files

  \begin{itemize}
  \tightlist
  \item
    type \texttt{?} to pull up a file, for example \texttt{?sum}
  \item
    for more extensive questions, Google is also useful!
  \end{itemize}
\end{itemize}
\end{frame}

\begin{frame}[fragile]{Packages}
\phantomsection\label{packages}
\begin{itemize}
\tightlist
\item
  base functions (Base R) are automatically installed with R

  \begin{itemize}
  \tightlist
  \item
    includes mathematical operations, data manipulation, plotting, etc.
  \end{itemize}
\item
  a package is a way to store files with code, documentation, and data,
  and let users download and use those files
\item
  the \texttt{tidyverse} is a suite of common data manipulation and
  visualization packages

  \begin{itemize}
  \tightlist
  \item
    includes \texttt{dplyr}, \texttt{ggplot}, among others
  \end{itemize}
\end{itemize}
\end{frame}

\begin{frame}[fragile]{Packages}
\phantomsection\label{packages-1}
\begin{itemize}
\tightlist
\item
  CRAN package repository has \textasciitilde20,000 packages

  \begin{itemize}
  \tightlist
  \item
    most packages available here
  \item
    install with \texttt{install.packages("package\_name")}
  \item
    load in each R session with \texttt{library(package\_name)}
  \end{itemize}
\item
  Bioconductor has \textasciitilde2,000 packages
\item
  anyone can make their own package (often available to download on
  GitHub)
\end{itemize}
\end{frame}

\begin{frame}[fragile]{R Markdown}
\phantomsection\label{r-markdown}
\begin{itemize}
\tightlist
\item
  in R scripts (.R), each line is evaluated unless it is a comment

  \begin{itemize}
  \tightlist
  \item
    \texttt{\#\ this\ is\ a\ comment}
  \end{itemize}
\item
  in R Markdown files (.Rmd), you can combine code, output, and text

  \begin{itemize}
  \tightlist
  \item
    code in ``chunks'', anything within chunk is evaluated
  \item
    anything outside of chunk is output as text
  \end{itemize}
\item
  when compiled or ``knit'' .Rmd files turn into HTML, PDF, slides,
  webpages, etc.
\end{itemize}
\end{frame}

\begin{frame}{This session}
\phantomsection\label{this-session}
\begin{itemize}
\tightlist
\item
  download the ``R.zip'' file from \ldots{} and upload it to your remote
  RStudio instance
\item
  work through the file ``intro-to-R.Rmd'' and the accompanying
  tutorials

  \begin{itemize}
  \tightlist
  \item
    this is self-paced, start where you think will be the most helpful
    to you
  \end{itemize}
\item
  put your stickie notes up with questions!
\end{itemize}
\end{frame}

\end{document}
